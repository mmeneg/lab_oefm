La configurazione iniziale e' stata cosi realizzata
\begin{itemize}
	\item regolazione del cannocchiale: messa a fuoco del crocifilo e messa a fuoco all'infinito del cannocchiale, quest'ultima tramite messa a fuoco di un oggetto (porta metallica di accesso alle scale) distante almeno 15 metri dalla posizione del connocchiale (scrivania docenti in fondo al corridoio di ingresso al laboratorio);
	\item regolazione del collimatore: messa a fuoco del collimatore ottenuta guardando nel cannocchiale e mettendo a fuoco l'immagine della luce emessa dalla lampada posta davanti al collimatore. Adeguamento della larghezza della fessura posta tra collimatore e lampada in modo da visualizzare una riga ben definita;
	\item messa in bolla della piattaforma: verifica che la piattaforma sia parallela al piano della superficie terrestre tramite una bolla posta sulla piattaforma e facendo ruotare la stessa.
\end{itemize}

La misura dell'angolo $\alpha$ al vertice del prisma e' stata eseguita ponendo il prisma sulla piattaforma e orientandolo in modo tale da far entrare il raggio riflesso dalla superficie del prisma (corrispondente ad un lato dell'angolo) all'interno del connocchiale posto in una qualsiasi posizione fissata. In tale configurazione viene letto l'angolo $\theta_1$ della posizione della piattaforma mobile rispetto al cannocchiale.
La piattafroma mobile e' stata poi ruotata in modo da far entrare il raggio riflesso dall'altra superficie del prisma (corrispondente all'altro lato dell'angolo) all'interno del cannocchiale (fisso rispetto alla posizione precedente). Viene cosi letto l'angolo $\theta_2$ della posizione della piattaforma mobile rispetto al cannocchiale.

La relazione che lega l'angolo $\alpha$ all'angolo $\Delta \theta = \theta_1 - \theta_2$ e' la seguente:
\[
	\Delta \theta = 180^{\circ} - \alpha
\]
da cui:
\[
	\alpha = 180^{\circ} - \Delta \theta
\]

Rimosso il prisma dalla piattaforma e' stato misurato l'angolo $\theta_0$ tra il cannocchiale e il collimatore quando il fascio di luce proveniente dalla lampada attraverso la fessura e' centrato nel crocifilo del cannocchiale.

Appoggiato e fissato nuovamente il prisma sulla piattaforma e' stato misurato l'angolo $\theta_{\lambda}$, relativo alla lunghezza d'onda $\lambda$, tra il cannocchiale ed il collimatore quando la luce corrispondente ad una determinata lunghezza d'onda $\lambda$ inverte il suo movimento durante la rotazione della piattaforma mobile.
Questo permette di ottenere l'angolo di deviazione minima $\delta_{min} (\lambda)$ tramite la relazione:
\[
	\delta_{min} (\lambda) = |\theta_{\lambda} - \theta_0|
\]

Note le lunghezze d'onda $\lambda$ delle componenti della luce emessa dalla lampada a mercurio, ricavata nella precedente esperienza "Spettrometro a Reticolo", e' possibile ricavare l'indice di rifrazione $\eta (\lambda)$ tramite la relazione:
\begin{align}
	\eta (\lambda) = 
	\frac{
		\sin{
			\left(
				\frac{
					\alpha + \sigma_{min} (\lambda)
				}{
					2
				}
			\right)
		}
	}
	{\sin{
		\left(
			\frac{\alpha}{2}
		\right)
		}
	}
\end{align}

Infine e' stata verificata la relazione di Cauchy che mette in relazione il quadrato dell'indice di rifrazione $\eta^2 (\lambda)$ con l'inverso del quadrato della lunghezza d'onda $1/ \lambda^2$ tramite la formula:
\begin{align}
	\eta^2 (\lambda) = a + \frac{b}{\lambda^2}
\end{align}
dove $a$ e $b$ sono delle costanti a valori reali.
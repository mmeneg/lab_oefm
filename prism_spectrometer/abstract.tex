\par\noindent\rule{\textwidth}{0.4pt}

\begin{abstract}
    \begin{center}
    %commenti finali
    %\par{\large{Lab report: Measuring the charge-to-mass ratio of the electron}}
    In questa esperienza sono stati calcolati gli indici di rifrazione $\eta (\lambda)$ di un prisma per diverse lunghezze d'onda $\lambda$ della luce emessa da un lampada ai vapori di mercurio.
    Tali valori sono stati utilizzati per verificare la relazione di Cauchy che mette in relazione diretta il quadrato degli indici di rifrazione $\eta^2(\lambda)$ ed il reciproco del quadrato della lunghezza d'onda $1/\lambda^2$.
    
    Sono state analizzate due prese dati da parte di gruppi diversi in giorni diversi.
    
    I dati del primo gruppo hanno evidenziato un errore sistematico tale da invalidare le misure.
    
    Mentre i dati del secondo gruppo hanno mostrato una buona correlazione tra la distribuzione osservata e la relazione di Cauchy.
    \end{center}
\end{abstract}

\par\noindent\rule{\textwidth}{0.4pt}
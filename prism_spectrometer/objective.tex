Scopo dell'esperienza e' determinare l'indice di rifrazione $\eta (\lambda)$ di un prisma, tramite la misura dell'angolo di deviazione minima di diverse frequenze $\lambda$, della luce emessa da una lampada a mercurio nell'attraversare il prisma stesso.

La misura degli angoli viene effettuta tramite lettura di una scala graduata angolare con sensibilita' $0.5^{\prime}$.

Per fare questo si determina l'angolo $\alpha$ corrispondente all'angolo al vertice del prisma, vengono poi misurati gli angoli di deflessione minima $\sigma_{min} (\lambda)$ di diverse lunghezze d'onda $\lambda$ di una luce emessa da una lampada a mercurio.

%L'indice di rifrazione $\eta (\lambda)$ del prisma e' determinato dalla relazione:
%\begin{align*}
%	\eta (\lambda) = \frac{\sin{\left(\frac{\alpha + \sigma_{min} (\lambda)}{2}}\right)}{\sin{\left(\frac{\alpha}{2}}\right)}
%\end{align*}

Viene inoltre verificata la relazione di Cauchy che mette in relazione di proporzionalita' diretta il quadrato dell'indice di rifrazione $\eta^2 (\lambda)$ con il reciproco del quadrato della lunghezza d'onda $1/ \lambda^2$
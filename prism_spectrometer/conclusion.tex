Per il gruppo A, l'errore nella presa dati dell'angolo di deviazione minima $\delta_{min}$ non ha permesso di compiere ulteriori analisi.
Questo errore può essere derivato da un errore nella procedura eseguita nelle misurazioni di $\theta_0$ e di $\theta_{\lambda}$.

Per il gruppo B, il disaccordo tra la distribuzione attesa e l'andamento approssimato di Cauchy tra indice di rifrazione $\eta$ e lunghezza d'onda $\lambda$ è altamente significativo e rigettiamo quindi l'ipotesi.

Tuttavia dall'analisi dei dati emergono alcuni elementi degni di nota, in particolare:

I valori del $\chi^2$ ci suggeriscono che le misure di $\delta_{min}$ della riga "giallo sx" possano contenere un errore sistematico probabilmente dovuto ad una difficolta' nella visualizzazione della riga a causa della vicinanza della riga vicina "giallo dx".

Non è stato presa in considerazione l'incertezza dovuta alla larghezza della riga che viene centrata nel crocifilo, tale incertezza e' presente in tutte le misure della posizione di devizione minima $\theta_{\lambda}$ per ogni lunghezza d'onda $\lambda$ e non diminuisce ripetendo le misure. Pertanto contribuisce all'incertezza come errore sistematico $(\sigma_{\theta})_s$ e avremmo dovuto sommarlo in quadratura all'errore $\sigma_{\delta}$. 

Per migliorare la procedura sperimentale potremmo prestare maggiore attenzione nella presa dati relativa alla riga "giallo sx", sia aumentando il numero delle misure e sia facendo determinare il punto di inversione e la lettura del nonio non dalla stessa persona e alternando le persone.
Inoltre dovremmo determinare la larghezza della riga centrata nel crocofilo in modo da considerare l'errore sistematico sull'angolo di deviazione minima.
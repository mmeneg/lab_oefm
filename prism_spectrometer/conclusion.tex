Per il gruppo A, l'errore nella presa dati dell'angolo di deviazione minima $\delta_{min}$ non ha permesso di compiere ulteriori analisi.

Per il gruppo B, l'andamento approssimato di Cauchy tra indice di rifrazione $\eta$ e lunghezza d'onda $\lambda$ e' rispettato con una buona approssimazione.
I valori del $\chi^2$ ci suggeriscono che le misure di $\delta_{min}$ della riga "giallo sx" possano contenere un errore sistematico probabilmente dovuto ad una difficolta' nella visualizzazione a causa della vicinanza dell'altra riga "giallo dx".

Considerati gli errori $\sigma_{\lambda}$ delle lunghezze d'onda $\lambda$ misurate nell'esperienza "spettromentro a reticolo", si e' deciso di non propagare tali errori ai parametri del fit poiche' di un ordine di grandezza inferiore all'errore $\sigma_{\eta^2}$ e quindi trascurabili.

La buona correlazione, anche se migliorabile, tra la distribuzione osservata e la relazione di Cauchy ci garantisce che gli errori sistematici propagati siano abbastanza piccoli.
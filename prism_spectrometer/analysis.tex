I dati che verranno di seguito analizzati sono stati raccolti, seguendo la medesima procedura descritta precedentemente, in un due momenti differenti dai seguenti gruppi:
\begin{itemize}
	\item Gruppo A: il 6 Novembre 2024, da S. Pini e V. Malinouskaya
	\item Gruppo B: il 12 Novembre 2024, da A. Spagnolo e M. Meneghini
\end{itemize}

\subsection{Gruppo A}
Di seguito sono riportate le misure degli angolo $\theta_1$ e $\theta_2$:
\begin{table}[!htbp]
    {\par\centering
    \begin{tabular}{ccccc}
        \hline
        Misura & $\theta_1 \text{ ($^{\circ}$)}$ & $\theta_1 \text{ ($^{\prime}$)}$ & $\theta_2 \text{ ($^{\circ}$)}$ & $\theta_2 \text{ ($^{\prime}$)}$ \\
        \hline
        1   &	354	&	20.0	&	234	&	20.0\\
        2	&	348	&	57.0	&	228	&	58.0\\
        3	&	353	&	19.0	&	233	&	19.0\\
        4	&	352	&	27.0	&	232	&	25.0\\
        5	&	349	&	2.0	&	229	&	5.0\\
        6	&	354	&	22.0	&	234	&	28.0\\
        7	&	359	&	40.0	&	239	&	40.0\\
        8	&	359	&	45.0	&	239	&	43.0\\
        9	&	352	&	20.0	&	232	&	18.0\\
        10	&	355	&	5.0	&	235	&	6.0\\
        \hline
    \end{tabular}
    \par}
    \caption{Gruppo A - misura degli angoli $\theta_1$ e $\theta_2$}
\end{table}

%Ogni misura degli angoli $\theta_1$ e $\theta_2$ ha una incertezza pari $0.5^{\prime}$, in gradi decimali $0.008^{\circ}$.

L'angolo $\alpha$ e' ottenuto tramite la formula:
\[
	\alpha = 180^{\circ} - \Delta \theta
\]

Di seguito i valori di $\Delta \theta$ e $\alpha$:
\begin{table}[!htbp]
    {\par\centering
    \begin{tabular}{ccccc}
        \hline
        Misura & $\Delta \theta \text{ ($^{\circ}$)}$ & $\Delta \theta \text{ ($^{\prime}$)}$ & $\alpha \text{ ($^{\circ}$)}$ & $\alpha \text{ ($^{\prime}$)}$\\
        \hline
        1   &   120 &   0.0     &   60  &   0.0\\
        2   &   119 &   59.0    &   60  &   1.0\\
        3   &   120 &   0.0     &   60  &   0.0\\
        4   &   120 &   2.0     &   59  &   58.0\\
        5   &   119 &   57.0    &   60  &   3.0\\
        6   &   119 &   54.0    &   60  &   6.0\\
        7   &   120 &   0.0     &   60  &   0.0\\
        8   &   120 &   2.0     &   59  &   58\\
        9   &   120 &   2.0     &   59  &   58\\
        10  &   119 &   59.0    &   60  &   1.0\\
        \hline
    \end{tabular}
    \par}
    \caption{Gruppo A - angoli $\Delta \theta$ e $\alpha$ calcolati}
\end{table}

Il miglior valore di $\alpha$ e' ottenuto tramite la media delle dieci misurazioni indipendenti effettuate, mentre l'errore $\sigma_{\alpha}$ e' ottenuto come deviazione standard della media,
\[
    \alpha = 60^{\circ} \ 0.5^{\prime} \pm 0.8^{\prime}
\]
%in gradi decimali
%\[
%    \alpha = (60.01 \pm 0.01)^{\circ}
%\]

Di seguito sono riportate le misure dell'angolo $\theta_0$ relativo alla posizione del cannocchiale rispetto al collimatore con il raggio non deviato:
\begin{table}[!htbp]
    {\par\centering
    \begin{tabular}{ccc}
        \hline
        Misura & $\theta_0 \text{ ($^{\circ}$)}$ & $\theta_0 \text{ ($^{\prime}$)}$ \\
        \hline
        1   &   238 &   52.0\\
        2   &   238 &   53.0\\
        3   &   238 &   53.0\\
        4   &   238 &   53.0\\
        5   &   238 &   51.0\\
        6   &   238 &   54.0\\
        7   &   238 &   52.0\\
        8   &   238 &   53.0\\
        9   &   238 &   53.0\\
        10  &   238 &   52.0\\
        \hline
    \end{tabular}
    \par}
    \caption{Gruppo A - misura angolo $\theta_0$}
\end{table}

Il miglior valore di $\theta_0$ e' ottenuto tramite la media delle dieci misurazioni indipendenti effettuate, mentre essendo la deviazione standard della media $\sigma_{\bar{\theta_0}}$ inferiore alla sensibilita' del nonio, viene preso come errore $\sigma_{\theta_0}$ la sensibilita' del nonio.

\[
    \theta_0 = 238^{\circ} \ 52.6^{\prime} \pm 0.5^{\prime}
\]

Di seguito e' riportata la misura dell'angolo $\theta_{\lambda}$ in corrispondenza di una determinata lunghezza d'onda $\lambda$:
\begin{table}[!htbp]
    {\par\centering
    \begin{tabular}{lccc}
        \hline
            Colore &
            $\lambda \text{(nm)}$ &
            $\theta_{\lambda} \text{ ($^{\circ}$)}$ & 
            $\theta_{\lambda} \text{ ($^{\prime}$)}$ \\
        \hline
        viola       &   404.2   &   130 &   46.0\\
        verde scuro &   491.4   &   127 &   42.0\\
        verde       &   547.7   &   126 &   3.0\\
        giallo dx   &   576.9   &   125 &   13.0\\
        rosso       &   690.8   &   123 &   8.0\\
        \hline
    \end{tabular}
    \par}
    \caption{Gruppo A - misura angolo $\theta_{\lambda}$ in funzione di $\lambda$}
\end{table}

L'incertezza $\sigma_{\theta_{\lambda}}$ sulla singola misura di $\theta_{\lambda}$ e' stata stimata pari a $1.0^{\prime}$ per la difficolta' nell'individuare nitidamente il punto esatto di inversione.

Il valore dell'angolo di deviazione minima $\delta_{\lambda}$ e' ottenuto tramite la relazione:
\[
    \delta_{min} (\lambda) = |\theta_{\lambda} - \theta_0|
\]

mentre l'incertezza $\sigma_{delta}$ e' pari alla somma in quadratura di $\sigma_{\theta_{\lambda}}$ e $\sigma_{\theta_0}$.

Di seguito e' riportato il valore calcolato di $\delta_{min}$ e la sua incertezza $\sigma_{delta}$ in funzione di $\lambda$:
\begin{table}[!htbp]
    {\par\centering
    \begin{tabular}{lcccc}
        \hline
            Colore&
            $\lambda \text{(nm)}$ & 
            $\delta_{min} \text{ ($^{\circ}$)}$ & 
            $\delta_{min} \text{ ($^{\prime}$)}$ & 
            $\sigma_{delta} \text{ ($^{\prime}$)}$\\
        \hline
            viola       &   404.2   &   108 &   6.6  & 1.1\\
            verde scuro &   491.4   &   111 &   10.6 & 1.1\\
            verde       &   547.7   &   112 &   49.6 & 1.1\\
            giallo dx   &   576.9   &   113 &   39.6 & 1.1\\
            rosso       &   690.8   &   115 &   44.6 & 1.1\\
        \hline
    \end{tabular}
    \par}
    \caption{Gruppo A - angolo $\delta_{min}$ in funzione di $\lambda$}
\end{table}

Il valore di $\eta (\lambda)$ e' stato calcolato tramite la relazione:
\begin{align}
    \eta (\lambda) = \frac{\sin{\frac{\alpha + \sigma_{min} (\lambda)}{2}}}{\sin{\frac{\alpha}{2}}}
\end{align}

mentre l'errore $\sigma_{\eta}$ e' stato calcolato tramite la formula di propagazione degli errori, ovvero la relazione:
\begin{align}
    \sigma_{\eta} = 
        \left|
            \frac{
                \cos{\left( \frac{\alpha + \sigma_{min}}{2} \right)}
            }{
                2 \sin{\left( \frac{\alpha}{2} \right)}
            }
        \right|
        \sqrt{
            \left(
                1-\frac{
                    \tan{\left(
                        \frac{\alpha + \sigma_{min}}{2}
                    \right)
                }}{
                \tan{\left(
                    \frac{\alpha}{2}
                \right)}}
            \right)^2 \sigma_{\alpha}^2 + \sigma_{\delta}^2}
\end{align}

Di seguito sono riportati i valori di $\eta$ e $\sigma_{\eta}$ calcolati:
\begin{table}[!htbp]
    {\par\centering
    \begin{tabular}{lccc}
        \hline
            Colore &
            $\lambda \text{(nm)}$ & 
            $\eta$ & 
            $\sigma_{\eta}$ \\
        \hline
        viola       &   404.2   &   1.9890 &   0.0003 \\
        verde scuro &   491.4   &   1.9938 &   0.0003 \\
        verde       &   547.7   &   1.9958 &   0.0003 \\
        giallo dx   &   576.9   &   1.9966 &   0.0003 \\
        \hline
    \end{tabular}
    \par}
    \caption{Gruppo A - indice di rifrazione $\eta$ in funzione di $\lambda$}
\end{table}

Come si nota osservando i dati, la relazione tra la lunghezza d'onda $\lambda$ e l'indice di rifrazione $\eta$ e' di proporzionalita' diretta per tutte le lunghezze d'onda, contrariamente a quanto ci si sarebbe aspettato.
A maggior ragione mettendo in relazione il quadrato dell'indice di rifrazione $\eta_{\lambda}^2$ con il quadrato del reciproco della lunghezza d'onda $1/\lambda^2$ notiamo che la relazione potrebbe essere di proporzionalità diretta ma con una pendenza negativa.

\begin{center}
\begin{tikzpicture}
    \begin{axis}[
        title=Andamento $\eta^2 (\lambda)$,
        xlabel=$1/\lambda^2 \text{ (nm$^{-2}$)}$,
        ylabel=$\eta^2$,
        minor y tick num=1,
        minor x tick num=1,
        x tick label style={
            /pgf/number format/.cd,
            fixed,
            fixed zerofill,
            precision=1,
            /tikz/.cd
        }
    ]
        \addplot+[scatter, only marks, mark=*, mark options={solid}, mark size=1.0pt, error bars/.cd, y dir=both,y explicit, error mark=-]
            coordinates {
                (0.00000612,3.9560)   +-  (0,0.0011)
                (0.00000414,3.9752)   +-  (0,0.0012)
                (0.00000333,3.9832)  +-  (0,0.0012)
                (0.00000300,3.9866)  +-  (0,0.0012)
        };
    \end{axis}
\end{tikzpicture}
\end{center}

Molto probabilmente le misure dell'angolo di deviazione minima sono state prese erroneamente, magari leggendo dal nonio sbagliato, di conseguenza tali misure non sono valide.

\subsection{Gruppo B}
Di seguito sono riportate le misure degli angolo $\theta_1$ e $\theta_2$:
\begin{table}[!htbp]
    {\par\centering
    \begin{tabular}{ccccc}
        \hline
            Misura & 
            $\theta_1 \text{ ($^{\circ}$)}$ &
            $\theta_1 \text{ ($^{\prime}$)}$ & 
            $\theta_2 \text{ ($^{\circ}$)}$ & 
            $\theta_2 \text{ ($^{\prime}$)}$ \\
        \hline
        1   &   246 &   30.0    &   7  &   30.0\\
        2   &   249 &   0.0     &   9  &   0.0\\
        3   &   256 &   42.0    &   16 &   41.0\\
        4   &   253 &   29.0    &   12 &   27.0\\
        5   &   242 &   0.0     &   2  &   0.0\\
        \hline
    \end{tabular}
    \par}
    \caption{Gruppo B - misura degli angoli $\theta_1$ e $\theta_2$}
\end{table}

Di seguito i valori $\Delta \theta$ calcolati per ogni singola misura espressi in gradi decimali:
\begin{table}[!htbp]
    {\par\centering
    \begin{tabular}{cc}
        \hline
        Misura & $\Delta \theta \text{ ($^{\circ}$)}$ \\
        \hline
        1   &   239.00 \\
        2   &   240.00 \\
        3   &   240.02 \\
        4   &   241.03 \\
        5   &   240.00 \\
        \hline
    \end{tabular}
    \par}
    \caption{Gruppo B - angoli $\Delta \theta$ calcolati}
\end{table}

Notiamo che due misure (la \#1 e la \#4) hanno valori di $\Delta \theta$ molto disallineati rispetto alle altre tre.
La media delle misure e' pari a $240.01^{\circ}$ e l'errore, ottenuto come deviazione standard della media, pari a $0.32^{\circ}$.
Osserviamo che le misure \#1 e \#4, si trovano al di fuori dell'intervallo di semiampiezza pari al triplo dell'errore e centrato sulla media, e vengono quindi rigettate.

Il miglior valore di $\Delta \theta$ e' quindi ottenuto come media delle tre misure rimanenti, mentre l'errore e' calcolato come semidispersione massima:

    \begin{align*}
        &\Delta \theta = (240.01 \pm 0.01)^{\circ} \\
        &\Delta \theta = 240^{\circ}\ 0.3^{\prime} \pm 0.5^{\prime} \\
    \end{align*}


L'angolo $\alpha$ e' quindi pari a:

    \begin{align*}
        &\alpha = (60.01 \pm 0.01)^{\circ} \\
        &\alpha = 60^{\circ} \ 0.3^{\prime} \pm 0.5^{\prime}
    \end{align*}


Di seguito sono riportate le misure dell'angolo $\theta_0$ relativo alla posizione del cannocchiale rispetto al collimatore con il raggio non deviato:
\begin{table}[!htbp]
    {\par\centering
    \begin{tabular}{ccc}
        \hline
        Misura & $\theta_0 \text{ ($^{\circ}$)}$ & $\theta_0 \text{ ($^{\prime}$)}$ \\
        \hline
        1   &   58 &   40.0\\
        2   &   58 &   40.5\\
        3   &   58 &   40.0\\
        4   &   58 &   40.0\\
        5   &   58 &   41.0\\
        6   &   58 &   41.0\\
        7   &   58 &   40.0\\
        8   &   58 &   41.0\\
        9   &   58 &   41.0\\
        10  &   58 &   40.5\\
        \hline
    \end{tabular}
    \par}
    \caption{Gruppo B - misura angolo $\theta_0$}
\end{table}

L'errore $\sigma_{\theta_0}$ calcolato come deviazione standard della media e' pari a $0.2^{\prime}$, quindi inferiore alla sensibilita' del nonio, viene quindi utilizzato come errore la sensibilita' del nonio.

Il valore di $\theta_0$ e' quindi:

    \begin{align*}
        &\theta_0 = (58.68 \pm 0.01)^{\circ} \\
        &\theta_0 = 58^{\circ} \ 40.5^{\prime} \pm 0.5^{\prime}
    \end{align*}


Di seguito la misura di $\theta_{\lambda}$, che corrisponde al punto di inversione del raggio rifratto dal prisma, in funzione della lunghezza d'onda $\lambda$ e relativo colore come rilevati nell'esperienza delle spettrometro a reticolo (la lunghezza d'onda del colore rosso non e' stata rilevata nell'esperienza precedente e viene quindi usato il valore tabulato assumendo una incertezza trascurabile):

\begin{table}[!htbp]
    {\par\centering
    \begin{tabular}{clcccrr}
        \hline
            Misura & 
            Colore & 
            $\lambda \text{ (nm) }$ & 
            $\sigma_{\lambda} \text{(nm)}$ & 
            $\theta_{\lambda} \text{ ($^{\circ}$)}$ & 
            $\theta_{\lambda} \text{ ($^{\prime}$)}$ & 
            $\sigma_{\theta_{\lambda}} \text{ ($^{\prime}$)}$ \\
        \hline
            1   &   verde        &   547.2  & 1.8 &  351 &   25.0  & 1.0  \\
            2   &   giallo sx    &   580.1  & 1.9 &  352 &   20.5  & 1.0  \\
            3   &   indaco       &   434.2  & 2.5 &  347 &   6.0   & 1.0  \\
            4   &   giallo dx    &   576.4  & 3.8 &  352 &   26.5  & 1.0  \\
            5   &   verde        &   547.2  & 1.8 &  351 &   20.0  & 1.0  \\
            6   &   indaco       &   434.2  & 2.5 &  347 &   0.0   & 1.0  \\
            7   &   verde scuro  &   491.4  & 2.8 &  349 &   45.0  & 1.0  \\
            8   &   giallo dx    &   576.4  & 3.8 &  352 &   4.5   & 1.0  \\
            9   &   giallo sx    &   580.1  & 1.9 &  352 &   32.0  & 1.0  \\
            10  &   verde scuro  &   491.4  & 2.8 &  349 &   40.0  & 1.0  \\
            11  &   giallo dx    &   576.4  & 3.8 &  352 &   12.0  & 1.0  \\
            12  &   giallo sx    &   580.1  & 1.9 &  352 &   20.0  & 1.0  \\
            13  &   verde        &   547.2  & 1.8 &  351 &   30.5  & 1.0  \\
            14  &   indaco       &   434.2  & 2.5 &  347 &   7.0   & 1.0  \\
            15  &   verde scuro  &   491.4  & 2.8 &  349 &   48.5  & 1.0  \\
            16  &   rosso        &   690.8  & 0.0 &  354 &   0.0   & 1.0  \\
            17  &   giallo dx    &   576.4  & 3.8 &  352 &   12.0  & 1.0  \\
            18  &   giallo sx    &   580.1  & 1.9 &  352 &   20.0  & 1.0  \\
            19  &   verde        &   547.2  & 1.8 &  351 &   8.5   & 1.0  \\
            20  &   indaco       &   434.2  & 2.5 &  347 &   0.0   & 1.0  \\
            21  &   verde scuro  &   491.4  & 2.8 &  349 &   47.5  & 1.0  \\
            22  &   rosso        &   690.8  & 0.0 &  354 &   0.0   & 1.0  \\
            23  &   giallo dx    &   576.4  & 3.8 &  352 &   11.0  & 1.0  \\
            24  &   giallo sx    &   580.1  & 1.9 &  352 &   20.0  & 1.0  \\
            25  &   verde        &   547.2  & 1.8 &  351 &   28.0  & 1.0  \\
            26  &   indaco       &   434.2  & 2.5 &  347 &   5.0   & 1.0  \\
        \hline
    \end{tabular}
    \par}
    \caption{Gruppo B - misura del punto di inversione $\theta_{\lambda}$}
\end{table}

Purtroppo durante le misure del punto di inversione non abbiamo valutato correttamente l'incertezza, qualitativamente legata alle difficoltà di inquadrare il crocifilo nel centro della riga, e quindi proporzionale alla larghezza stessa della riga.
Abbiamo quindi assunto l'incertezza $\sigma_{\theta_{\lambda}}$ della misura corrispondente al punto di inversione come il doppio della sensibilità del nonio, ovvero $1^{\prime}$. 

L'errore $\sigma_{\delta}$ dell'angolo di deviazione minima e' calcolato come la somma in quadratura di $\sigma_{\theta_0}$ e $\sigma_{\theta_{\lambda}}$.

L'errore $\sigma_{\eta}$ e' calcolato mediante la formula di propagazione dell'errore.

Di seguito i valori dell'angolo di deviazione minima $\delta_{min}$, $\sigma_{\delta}$, l'indice di rifrazione $\eta$, $\sigma_{\eta}$:

\begin{table}[!htbp]
    {\par\centering
    \begin{tabular}{clcccrcc}
        \hline
            Misura &
            Colore & 
            $\lambda \text{ (nm) }$ & 
            $\sigma_{\lambda} \text{(nm)}$ & 
            $\delta_{min} \text{ ($^{\circ}$)}$ & 
            $\sigma_{\delta} \text{ ($^{\circ}$)}$ & 
            $\eta$ &
            $\sigma_{\eta}$ \\
        \hline
        1   &   verde        &   547.2  & 1.8 &  67.26 &   0.02  & 1.7918 & 0.0002  \\
        2   &   giallo sx    &   580.1  & 1.9 &  66.33 &   0.02  & 1.7845 & 0.0002  \\
        3   &   indaco       &   434.2  & 2.5 &  71.58 &   0.02  & 1.8239 & 0.0002  \\
        4   &   giallo dx    &   576.4  & 3.8 &  66.23 &   0.02  & 1.7838 & 0.0002  \\
        5   &   verde        &   547.2  & 1.8 &  67.34 &   0.02  & 1.7924 & 0.0002  \\
        6   &   indaco       &   434.2  & 2.5 &  71.68 &   0.02  & 1.8247 & 0.0002  \\
        7   &   verde scuro  &   491.4  & 2.8 &  68.93 &   0.02  & 1.8045 & 0.0002  \\
        8   &   giallo dx    &   576.4  & 3.8 &  66.60 &   0.02  & 1.7866 & 0.0002  \\
        9   &   giallo sx    &   580.1  & 1.9 &  66.14 &   0.02  & 1.7830 & 0.0002  \\
        10  &   verde scuro  &   491.4  & 2.8 &  69.01 &   0.02  & 1.8051 & 0.0002  \\
        11  &   giallo dx    &   576.4  & 3.8 &  66.48 &   0.02  & 1.7857 & 0.0002  \\
        12  &   giallo sx    &   580.1  & 1.9 &  66.34 &   0.02  & 1.7846 & 0.0002  \\
        13  &   verde        &   547.2  & 1.8 &  67.17 &   0.02  & 1.7911 & 0.0002  \\
        14  &   indaco       &   434.2  & 2.5 &  71.56 &   0.02  & 1.8238 & 0.0002  \\
        15  &   verde scuro  &   491.4  & 2.8 &  68.87 &   0.02  & 1.8041 & 0.0002  \\
        16  &   rosso        &   690.8  & 0.0 &  64.68 &   0.02  & 1.7713 & 0.0002  \\
        17  &   giallo dx    &   576.4  & 3.8 &  66.48 &   0.02  & 1.7857 & 0.0002  \\
        18  &   giallo sx    &   580.1  & 1.9 &  66.34 &   0.02  & 1.7846 & 0.0002  \\
        19  &   verde        &   547.2  & 1.8 &  67.53 &   0.02  & 1.7939 & 0.0002  \\
        20  &   indaco       &   434.2  & 2.5 &  71.68 &   0.02  & 1.8247 & 0.0002  \\
        21  &   verde scuro  &   491.4  & 2.8 &  68.88 &   0.02  & 1.8042 & 0.0002  \\
        22  &   rosso        &   690.8  & 0.0 &  64.68 &   0.02  & 1.7713 & 0.0002  \\
        23  &   giallo dx    &   576.4  & 3.8 &  66.49 &   0.02  & 1.7858 & 0.0002  \\
        24  &   giallo sx    &   580.1  & 1.9 &  66.34 &   0.02  & 1.7846 & 0.0002  \\
        25  &   verde        &   547.2  & 1.8 &  67.21 &   0.02  & 1.7914 & 0.0002  \\
        26  &   indaco       &   434.2  & 2.5 &  71.59 &   0.02  & 1.8241 & 0.0002  \\
        \hline
    \end{tabular}
    \par}
    \caption{Gruppo B - misura del punto di inversione $\theta_{\lambda}$}
\end{table}

Per ogni singola lunghezza d'onda avendo a disposizione cinque misure (ad eccezione del verde scuro: quattro misure, e del rosso: due misure), l'errore $\sigma_{\eta}$ viene preso come errore massimo per compensare l'errore sottostimato nella misura di $\delta_{min}$ e calcolato come semidispersione massima.

Fa eccezione il colore rosso, per il quale abbiamo disponibili due misure identiche, per questo viene preso come errore $\sigma_{\eta}$ il triplo dell'errore propagato, a significare che la vera misura di $\eta_{rosso}$ ha il 99.7\% di probabilità di trovarsi entro tale intervallo.

\begin{table}[!htbp]
    {\par\centering
    \begin{tabular}{lcccc}
        \hline
            Colore & 
            $\lambda \text{ (nm) }$ &
            $\sigma_{\lambda} \text{(nm)}$ & 
            $\eta$ &
            $\sigma_{\eta}$ \\
        \hline
        verde        &   547.2  & 1.8 &  1.7921 &   0.0014 \\
        giallo sx    &   580.1  & 1.9 &  1.7843 &   0.0008 \\
        indaco       &   434.2  & 2.5 &  1.8242 &   0.0004 \\
        giallo dx    &   576.4  & 3.8 &  1.7855 &   0.0014 \\
        verde scuro  &   491.4  & 2.8 &  1.8045 &   0.0005 \\
        rosso        &   690.8  & 0.0 &  1.7713 &   0.0006 \\
        \hline
    \end{tabular}
    \par}
    \caption{Gruppo B - coefficiente di rifrazione $\eta$ in funzione di $\lambda$}
\end{table}

Per verificare la relazione di Cauchy sono stati messi in relazione i valori di $\eta^2$ e $1/\lambda^2$
e valutando il fit tramite regressione lineare.
%Per riportare l'errore $\sigma_{\lambda}$ sull'asse di $\eta^2$ è stato valutato il fattore $\alpha_{test}$ tra le lunghezze d'onda agli estremi (rosso e indaco)
%\[
%    \alpha_{test} = 5.9 \cdot 10^4 \ \text{nm$^2$}
%\]

Di seguito i valori utilizzati per la regressione:
\begin{table}[!htbp]
    {\par\centering
    \begin{tabular}{lccc}
        \hline
            Colore &
            $\eta^2$ &
            $1/\lambda^2 \text{ (nm$^{-2}$)}$ &
            $\sigma_{\eta^2}$ \\%&
            %$\sigma_{1/\lambda^2} \text{ (nm$^{-2}$)}$ &
            %$\sigma_{\eta^2} \text{ da } \sigma_{1/\lambda^2}$ &
            %$\sigma_{\eta^2} \text{ TOT}$ \\
        \hline
        verde        &  3.212   &   3.33$\cdot$10$^{-6}$  & 0.005 \\%&  2.20$\cdot$10$^{-8}$ &   0.001 & 0.002\\
        giallo sx    &  3.184   &   2.97$\cdot$10$^{-6}$  & 0.003 \\%&  1.95$\cdot$10$^{-8}$ &   0.001 & 0.002\\
        indaco       &  3.328   &   5.29$\cdot$10$^{-6}$  & 0.002 \\%&  6.11$\cdot$10$^{-8}$ &   0.004 & 0.004\\
        giallo dx    &  3.188   &   3.00$\cdot$10$^{-6}$  & 0.005 \\%&  3.97$\cdot$10$^{-8}$ &   0.002 & 0.003\\
        verde scuro  &  3.256   &   4.14$\cdot$10$^{-6}$  & 0.002 \\%&  4.72$\cdot$10$^{-8}$ &   0.003 & 0.003\\
        rosso        &  3.138   &   2.10$\cdot$10$^{-6}$  & 0.002 \\%&  0                    &   0     & 0.001\\
        \hline
    \end{tabular}
    \par}
    \caption{Gruppo B - valori utilizzati per la regressione lineare}
\end{table}

L'errore $\sigma_{\eta^2}$ e' calcolato tramite propagazione dell'errore:
\begin{align}
    \sigma_{\eta^2} = 2 \eta \sigma_{\eta}
\end{align}

Di seguito il grafico della regressione lineare:
\begin{center}
\begin{tikzpicture}
    \begin{axis}[
        title=Andamento $\eta^2 (\lambda)$,
        xlabel=$1/\lambda^2 \text{ (nm$^{-2}$)}$,
        ylabel=$\eta^2$,
        minor y tick num=1,
        minor x tick num=1,
        x tick label style={
            /pgf/number format/.cd,
            fixed,
            fixed zerofill,
            precision=1,
            /tikz/.cd
        }
    ]
        \addplot[color=blue, domain=0.0000020:0.0000055]{3.00921+x*60011.682180};
        \addplot+[scatter, only marks, mark=*, mark options={solid}, mark size=1.0pt, error bars/.cd, y dir=both,y explicit, error mark=-]
            coordinates {
                (0.00000333,3.2116)   +-  (0,0.0051)
                (0.00000297,3.1836)   +-  (0,0.0028)
                (0.00000529,3.3278)  +-  (0,0.0015)
                (0.00000300,3.1880)  +-  (0,0.0051)
                (0.00000414,3.2561)  +-  (0,0.0019)
                (0.00000210,3.1375)  +-  (0,0.0023)
        };
    \end{axis}
\end{tikzpicture}
\end{center}

La regressione lineare fornisce le seguenti stime dei coefficienti $a$ e $b$:
\begin{align*}
    &b = (6.00 \pm 0.08) \cdot 10^4 \text{ nm$^2$} \\
    &a = (3.009 \pm 0.003)
\end{align*}

Eseguendo il test del "chi quadro" otteniamo i seguenti risultati:
\begin{align*}
    &\chi_{tot}^2 = 5.3\\
    &\text{GdL} = 4 \\
    &\tilde{\chi}^2= 1.3 \\
\end{align*}
che corrisponde ad una probabilità del 26\%, a significare che la distribuzione osservata e' descrivibile con la relazione di Cauchy.

Inoltre notiamo che la misura relativa alla riga del giallo di sinistra, "giallo sx", ha un "chi quadro" molto alto rispetto alle altre misure, come evidenziato dalla tabella sottostante:
\begin{table}[!htbp]
    {\par\centering
    \begin{tabular}{cccccc}
        \hline
            Colore &
            $\eta^2$ &
            $1/\lambda^2 \text{ (nm$^{-2}$)}$ &
            $\sigma_{\eta^2}$ &
            $\eta^2 \text{ da modello}$ &
            $\chi^2$\\
        \hline
        verde       &   3.212   &   3.33$\cdot$10$^{-6}$  & 0.005   &   3.210   &   0.1\\
        giallo sx   &   3.184   &   2.97$\cdot$10$^{-6}$  & 0.003   &   3.188   &   2.9\\
        indaco      &   3.328   &   5.29$\cdot$10$^{-6}$  & 0.002   &   3.326   &   0.9\\
        giallo dx   &   3.188   &   3.00$\cdot$10$^{-6}$  & 0.005   &   3.191   &   0.3\\
        verde scuro &   3.256   &   4.14$\cdot$10$^{-6}$  & 0.002   &   3.258   &   1.1\\
        rosso       &   3.137   &   2.10$\cdot$10$^{-6}$  & 0.001   &   3.137   &   0.1\\
        \hline
    \end{tabular}
    \par}
    \caption{Gruppo B - test del $\chi^2$}
\end{table}

%Questo ci suggerisce che le misure dell'angolo di devizione minima per la riga "giallo sx" 
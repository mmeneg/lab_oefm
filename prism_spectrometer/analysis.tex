I dati che verranno di seguito analizzati sono stati raccolti, seguendo la medesima procedura descritta precedentemente, in un due momenti differenti dai seguenti gruppi:
\begin{itemize}
	\item Gruppo A: il 6 Novembre 2024, da S. Pini e V. Malinouskaya
	\item Gruppo B: il 12 Novembre 2024, da A. Spagnolo e M. Meneghini
\end{itemize}

\subsection{Gruppo A}
Di seguito sono riportate le misure degli angolo $\theta_1$ e $\theta_2$:
\begin{table}[!htbp]
    {\par\centering
    \begin{tabular}{ccccc}
        \hline
        Misura & $\theta_1 \text{ ($^{\circ}$)}$ & $\theta_1 \text{ ($^{\prime}$)}$ & $\theta_2 \text{ ($^{\circ}$)}$ & $\theta_2 \text{ ($^{\prime}$)}$ \\
        \hline
        1   &	354	&	20.0	&	234	&	20.0\\
        2	&	348	&	57.0	&	228	&	58.0\\
        3	&	353	&	19.0	&	233	&	19.0\\
        4	&	352	&	27.0	&	232	&	25.0\\
        5	&	349	&	2.0	&	229	&	5.0\\
        6	&	354	&	22.0	&	234	&	28.0\\
        7	&	359	&	40.0	&	239	&	40.0\\
        8	&	359	&	45.0	&	239	&	43.0\\
        9	&	352	&	20.0	&	232	&	18.0\\
        10	&	355	&	5.0	&	235	&	6.0\\
        \hline
    \end{tabular}
    \par}
    \caption{Gruppo A - misura degli angoli $\theta_1$ e $\theta_2$}
\end{table}

%Ogni misura degli angoli $\theta_1$ e $\theta_2$ ha una incertezza pari $0.5^{\prime}$, in gradi decimali $0.008^{\circ}$.

L'angolo $\alpha$ e' ottenuto tramite la formula:
\[
	\alpha = 180^{\circ} - \Delta \theta
\]

Di seguito i valori di $\Delta \theta$ e $\alpha$:
\begin{table}[!htbp]
    {\par\centering
    \begin{tabular}{ccccc}
        \hline
        Misura & $\Delta \theta \text{ ($^{\circ}$)}$ & $\Delta \theta \text{ ($^{\prime}$)}$ & $\alpha \text{ ($^{\circ}$)}$ & $\alpha \text{ ($^{\prime}$)}$\\
        \hline
        1   &   120 &   0.0     &   60  &   0.0\\
        2   &   119 &   59.0    &   60  &   1.0\\
        3   &   120 &   0.0     &   60  &   0.0\\
        4   &   120 &   2.0     &   59  &   58.0\\
        5   &   119 &   57.0    &   60  &   3.0\\
        6   &   119 &   54.0    &   60  &   6.0\\
        7   &   120 &   0.0     &   60  &   0.0\\
        8   &   120 &   2.0     &   59  &   58\\
        9   &   120 &   2.0     &   59  &   58\\
        10  &   119 &   59.0    &   60  &   1.0\\
        \hline
    \end{tabular}
    \par}
    \caption{Gruppo A - angoli $\Delta \theta$ e $\alpha$ calcolati}
\end{table}

Il miglior valore di $\alpha$ e' ottenuto tramite la media delle dieci misurazioni indipendenti effettuate, mentre l'errore $\sigma_{\alpha}$ e' ottenuto come deviazione standard della media,
\[
    \alpha = 60^{\circ} \ 0.5^{\prime} \pm 0.8^{\prime}
\]
%in gradi decimali
%\[
%    \alpha = (60.01 \pm 0.01)^{\circ}
%\]

Di seguito sono riportate le misure dell'angolo $\theta_0$ relativo alla posizione del cannocchiale rispetto al collimatore con il raggio non deviato:
\begin{table}[!htbp]
    {\par\centering
    \begin{tabular}{ccc}
        \hline
        Misura & $\theta_0 \text{ ($^{\circ}$)}$ & $\theta_0 \text{ ($^{\prime}$)}$ \\
        \hline
        1   &   238 &   52.0\\
        2   &   238 &   53.0\\
        3   &   238 &   53.0\\
        4   &   238 &   53.0\\
        5   &   238 &   51.0\\
        6   &   238 &   54.0\\
        7   &   238 &   52.0\\
        8   &   238 &   53.0\\
        9   &   238 &   53.0\\
        10  &   238 &   52.0\\
        \hline
    \end{tabular}
    \par}
    \caption{Gruppo A - misura angolo $\theta_0$}
\end{table}

Il miglior valore di $\theta_0$ e' ottenuto tramite la media delle dieci misurazioni indipendenti effettuate, mentre essendo la deviazione standard della media $\sigma_{\bar{\theta_0}}$ inferiore alla sensibilita' del nonio, viene preso come errore $\sigma_{\theta_0}$ il doppio della deviazione standard della media $2\sigma_{\bar{\theta_0}}$,
\[
    \theta_0 = 238^{\circ} \ 52.6^{\prime} \pm 0.5^{\prime}
\]
a significare che il valore vero di $\theta_0$ ha il 95.5\% di probabilita' di essere contenuto in tale intervallo.

Di seguito e' riportata la misura dell'angolo $\theta_{\lambda}$ in corrispondenza di una determinata lunghezza d'onda $\lambda$:
\begin{table}[!htbp]
    {\par\centering
    \begin{tabular}{ccc}
        \hline
        $\lambda \text{(nm)}$ & $\theta_{\lambda} \text{ ($^{\circ}$)}$ & $\theta_{\lambda} \text{ ($^{\prime}$)}$ \\
        \hline
        404.7   &   130 &   46.0\\
        496.0   &   127 &   42.0\\
        546.1   &   126 &   3.0\\
        577.0   &   125 &   13.0\\
        NA   &   123 &   8.0\\
        \hline
    \end{tabular}
    \par}
    \caption{Gruppo A - misura angolo $\theta_{\lambda}$ in funzione di $\lambda$}
\end{table}

L'incertezza $\sigma_{\theta_{\lambda}}$ sulla singola misura di $\theta_{\lambda}$ e' stata stimata pari a $1.0^{\prime}$ per la difficolta' nell'individuare nitidamente il punto esatto di inversione.

Il valore dell'angolo di deviazione minima $\delta_{\lambda}$ e' ottenuto tramite la relazione:
\[
    \delta_{min} (\lambda) = |\theta_{\lambda} - \theta_0|
\]

mentre l'incertezza $\sigma_{delta}$ e' pari alla somma in quadratura di $\sigma_{\theta_{\lambda}}$ e $\sigma_{\theta_0}$.

Di seguito e' riportato il valore calcolato di $\delta_{min}$ e la sua incertezza $\sigma_{delta}$ in funzione di $\lambda$:
\begin{table}[!htbp]
    {\par\centering
    \begin{tabular}{cccc}
        \hline
        $\lambda \text{(nm)}$ & $\delta_{min} \text{ ($^{\circ}$)}$ & $\delta_{min} \text{ ($^{\prime}$)}$ & $\sigma_{delta} \text{ ($^{\prime}$)}$\\
        \hline
        404.7   &   108 &   6.6  & 1.1\\
        496.0   &   111 &   10.6 & 1.1\\
        546.1   &   112 &   49.6 & 1.1\\
        577.0   &   113 &   39.6 & 1.1\\
        NA   &   115 &   44.6 & 1.1\\
        \hline
    \end{tabular}
    \par}
    \caption{Gruppo A - angolo $\delta_{min}$ in funzione di $\lambda$}
\end{table}

Il valore di $\eta (\lambda)$ e' stato calcolato tramite la relazione:
\[
    \eta (\lambda) = \frac{\sin{\frac{\alpha + \sigma_{min} (\lambda)}{2}}}{\sin{\frac{\alpha}{2}}}
\]

mentre l'errore $\sigma_{\eta}$ e' stato calcolato tramite la formula di propagazione degli errori, ovvero la relazione:
\[
    %\sigma_{\eta} = \sqrt{\left( \frac{1}{2 \sin^2{(\frac{\alpha}{2}})} \left( \cos{\left(\frac{\alpha+\sigma_{min}}{2} \right)}\sin{\left(\frac{\alpha}{2}\right)} - \sin{\left( \frac{\alpha+\sigma_{min}}{2} \right)}\cos{\left(\frac{\alpha}{2}\right)} \right) \sigma_{\alpha} \right)^2 + \left( \frac{1}{2} \frac{\cos{\left(\frac{\alpha+\sigma_{min}}{2}}{\sin{\left(\frac{\alpha}{2}\right)}} \sigma_{\lambda} \right)^2}
    \sigma_{\eta} = \left| {\frac{\cos{\left( \frac{\alpha + \sigma_{min}}{2} \right)}}{2 \sin{\left(\frac{\alpha}{2}} \right)}} \right | \sqrt{\left( 1-\frac{\tan{\left(\frac{\alpha + \sigma_{min}}{2}}\right)}{\tan{\left( \frac{\alpha}{2} \right)}} \right)^2 \sigma_{\alpha}^2 + \sigma_{\delta}^2}
\]

Di seguito

\subsection{Gruppo B}
Di seguito sono riportate le misure degli angolo $\theta_1$ e $\theta_2$:
\begin{table}[!htbp]
    {\par\centering
    \begin{tabular}{ccccc}
        \hline
        Misura & $\theta_1 \text{ ($^{\circ}$)}$ & $\theta_1 \text{ ($^{\prime}$)}$ & $\theta_2 \text{ ($^{\circ}$)}$ & $\theta_2 \text{ ($^{\prime}$)}$ \\
        \hline
        1   &   246 &   30.0    &   7 &   30.0\\
        2   &   249 &   0.0    &   9 &   0.0\\
        3   &   256 &   42.0    &   16 &   41.0\\
        4   &   253 &   29.0    &   12 &   27.0\\
        5   &   242 &   0.0 &   2 &   0.0\\
        \hline
    \end{tabular}
    \par}
    \caption{Gruppo B - misura degli angoli $\theta_1$ e $\theta_2$}
\end{table}

L'angolo $\alpha$ e' ottenuto tramite la formula:
\[
    \alpha = 180^{\circ} - \Delta \theta
\]
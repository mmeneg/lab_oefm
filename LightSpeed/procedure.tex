
The experiment begins by determining three key distances in the rotating-mirror setup:
1. The distance between the rotating mirror and the concave mirror
\[
D = (13.42\pm\,0.03)\text{ m}
\]  
2. The distance between lens \(L_2\) and the rotating mirror
\[
a = ( 0.489 \pm\,0.002)\text{ m}
\]  
3. The focal length of lens \(L_2\)  
\[
f_{2} = ( 0.252 \pm\,0.001)\text{ m}
\]  

These distances enter the theoretical formula for calculating the speed of light. The rotating mirror is then aligned so that the laser beam strikes its center and subsequently reflects off two plane mirrors onto the concave mirror. Small micrometric adjustments ensure that the returning beam exactly retraces its path and exits partially through a beam splitter into a microscope. A micrometer in the microscope allows precise measurement (\(\pm 0.01\,\mathrm{mm}\)) of the lateral displacement \(\delta\).

After confirming alignment, the rotating mirror’s motor is switched on at a low reference frequency \(\nu_0\). The micrometer reading \(\delta_0\) at this initial speed provides a baseline spot position. The mirror’s frequency is then increased to a new value \(\nu\), and the spot’s position is re-centered in the microscope, yielding a new reading \(\delta\). The difference \(\Delta \delta = |\delta - \delta_0|\)  represents the displacement caused by the increased mirror rotation. Simultaneously, the difference in angular velocity is computed as \(\Delta \omega = |\omega - \omega_0|\), where \(\omega = 2\pi\,\nu\) and \(\omega_0 = 2\pi\,\nu_0\). We measure \(\Delta\delta\) and \(\Delta\omega\) as the absolute differences in displacement and angular velocity, since we only need the magnitude of the shift to apply the Foucault formula.

Three tables of measurements are recorded by GroupA, corresponding to:

1. Clockwise rotation at varying speeds

2. Counterclockwise rotation at varying speeds

3. Transitions from maximum speed in one direction to maximum speed in the other (“From max to max”).  

In each table, the following columns are included: \(\nu_{0}\) \((\mathrm{Hz})\): the initial (low) frequency, \(\nu\) \((\mathrm{Hz})\): the final frequency, \(\delta_{0}\) \((\mathrm{mm})\): the baseline micrometer reading at \(\nu_{0}\), \(\delta\) \((\mathrm{mm})\): the micrometer reading at \(\nu\),  \(\Delta \delta\) \((\mu\mathrm{m})\): the difference \(|\delta - \delta_{0}|\), expressed in micrometers, \(\Delta \omega\) \((\mathrm{rad/s})\): the difference \(|\omega - \omega_{0}|\).
Each individual measurement of \(c\) is obtained through the Foucault formula that uses \(\Delta \omega\) and \(\Delta \delta\):

\[
c 
\;=\; 
\frac{4\,f_{2}\,D^{2}\,\Delta\omega}
     {\bigl(D + a - f_{2}\bigr)\,\Delta \delta},
\]  



\[
\textbf{Table 1: Clockwise } 
\]
\[
\begin{array}{c|c|c|c|c|c|c|c}
\text{} 
& v_{0}\,(\text{Hz})
& v\,(\text{Hz})
& \delta_{0}\,(\mathrm{mm})
& \delta\,(\mathrm{mm})
& \Delta \delta\,(\mathrm{mm})
& \Delta \omega\,(\mathrm{rad/s})
& c\,(\mathrm{m/s})
\\ \hline
1 & 0 & 147.0 & 7.31 & 7.35 & 0.04 & 923.63 & 3.07\times10^{8} \\
2 & 147.0 & 1110.0 & 7.03 & 7.29 & 0.26 & 6050.71 & 3.09\times10^{8} \\
3 & 179.0 & 1111.0 & 7.04 & 7.29 & 0.25 & 5855.93 & 3.11\times10^{8} \\
4 & 263.0 & 1114.5 & 7.03 & 7.26 & 0.23 & 5350.13 & 3.09\times10^{8} \\
5 & 272.5 & 1110.5 & 7.02 & 7.27 & 0.25 & 5265.31 & 2.80\times10^{8} \\
6 & 56.0 & 1111.0 & 7.32 & 7.02 & 0.30 & 6628.76 & 2.94\times10^{8} \\
7 & 41.5 & 1113.5 & 7.03 & 7.32 & 0.29 & 6735.58 & 3.09\times10^{8} \\
8 & 54.5 & 1119.0 & 7.01 & 7.29 & 0.28 & 6688.45 & 3.18\times10^{8} \\
9 & 121.5 & 1121.0 & 6.75 & 7.01 & 0.26 & 6280.04 & 3.21\times10^{8} \\
10 & 60.0 & 1493.0 & 6.32 & 6.69 & 0.37 & 9003.81 & 3.23\times10^{8} \\
11 & 81.5 & 1493.0 & 6.61 & 6.98 & 0.37 & 8868.72 & 3.19\times10^{8} \\
12 & 94.5 & 1125.0 & 6.70 & 6.97 & 0.27 & 6474.82 & 3.19\times10^{8} \\
13 & 44.5 & 1120.0 & 6.34 & 6.08 & 0.26 & 6757.57 & 3.45\times10^{8} \\
14 & 70.5 & 1125.0 & 6.73 & 7.03 & 0.30 & 6625.62 & 2.94\times10^{8} \\
15 & 198.5 & 1122.0 & 6.70 & 6.95 & 0.25 & 5802.52 & 3.09\times10^{8} \\
\end{array}
\]

\[
\textbf{Table 2: Counterclockwise } 
\]
\[
\begin{array}{c|c|c|c|c|c|c|c}
\text{} 
& v_{0}\,(\text{Hz})
& v\,(\text{Hz})
& \delta_{0}\,(\mathrm{mm})
& \delta\,(\mathrm{mm})
& \Delta \delta\,(\mathrm{mm})
& \Delta \omega\,(\mathrm{rad/s})
& c\,(\mathrm{m/s})
\\ \hline
1 & 90.0 & 1119.5 & 7.34 & 7.07 & 0.27 & 6468.54 & 3.18\times10^{8} \\
2 & 151.0 & 1122.0 & 7.30 & 7.03 & 0.27 & 6100.97 & 3.00\times10^{8} \\
3 & 112.5 & 1121.0 & 6.92 & 6.65 & 0.27 & 6336.59 & 3.12\times10^{8} \\
4 & 40.0 & 1125.0 & 7.28 & 6.99 & 0.29 & 6817.26 & 3.12\times10^{8} \\
5 & 47.0 & 1125.0 & 6.81 & 6.53 & 0.28 & 6773.27 & 3.22\times10^{8} \\
6 & 70.5 & 1127.0 & 6.94 & 6.68 & 0.26 & 6638.19 & 3.39\times10^{8} \\
7 & 78.0 & 1123.0 & 6.85 & 6.59 & 0.26 & 6565.93 & 3.36\times10^{8} \\
8 & 87.0 & 1125.0 & 6.85 & 6.57 & 0.28 & 6521.95 & 3.10\times10^{8} \\
9 & 47.5 & 1125.0 & 6.86 & 6.58 & 0.28 & 6770.13 & 3.21\times10^{8} \\
10 & 266.0 & 1124.0 & 6.93 & 6.71 & 0.22 & 5390.97 & 3.26\times10^{8} \\
\end{array}
\]

\[
\textbf{Table 3: "From max to max" } 
\]
\[
\begin{array}{c|c|c|c|c|c|c|c}
\text{} 
& v_{0}\,(\text{Hz})
& v\,(\text{Hz})
& \delta_{0}\,(\mathrm{mm})
& \delta\,(\mathrm{mm})
& \Delta \delta\,(\mathrm{mm})
& \Delta \omega\,(\mathrm{rad/s})
& c\,(\mathrm{m/s})
\\ \hline
1 & -1114.0 & 1116.0 & 6.28 & 6.87 & 0.59 & 14011.50 & 3.16\times10^{8} \\
2 & -1119.0 & 1119.0 & 6.36 & 6.92 & 0.56 & 14061.77 & 3.34\times10^{8} \\
3 & -1121.0 & 1118.0 & 6.32 & 6.90 & 0.58 & 14068.05 & 3.22\times10^{8} \\
4 & -1121.0 & 1122.0 & 6.24 & 6.92 & 0.68 & 14093.18 & 2.75\times10^{8} \\
5 & -1120.0 & 1121.0 & 6.29 & 6.84 & 0.55 & 14080.62 & 3.40\times10^{8} \\
6 & -1120.0 & 1115.0 & 6.30 & 6.92 & 0.62 & 14042.92 & 3.01\times10^{8} \\
7 & -1119.0 & 1118.0 & 6.33 & 6.91 & 0.58 & 14055.49 & 3.22\times10^{8} \\
8 & -1119.0 & 1119.0 & 6.37 & 6.96 & 0.59 & 14061.77 & 3.17\times10^{8} \\
9 & -1119.0 & 1116.0 & 6.28 & 6.87 & 0.59 & 14042.92 & 3.16\times10^{8} \\
10 & -1121.0 & 1121.0 & 6.29 & 6.92 & 0.63 & 14086.90 & 2.97\times10^{8} \\
\end{array}
\]

GroupB repeats the same procedure and measurement strategy: one for clockwise rotation, one for counterclockwise, and one for transitions from maximum speed in one direction to maximum speed in the other.


TABLEEEEEEES

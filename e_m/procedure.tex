%Raccolta dati misurati sotto forma di tabella ordinata
%Presentare non solo i dati elaborati ma anche quelli grezzi
%(questi ultimi, eventualmente, in appendice).
%Presentare eventuali grafici con gli assi correttamente definiti (unità di 
%misura, cifre significative, identificativo assi).


%Sample: Table
%\begin{table}[!htbp]
%    {\par\centering
%    \begin{tabular}{ccccc}
%        \hline
%        Misura & $M \text{ (g)}$ & $\sigma_{M} \text{ (g)}$ & $L \text{ (cm)}$ & $\sigma_{L} \text{ (cm)}$ \\
%        \hline
%        1   &   324.2&   0.1 &   79.5    &   0.1\\
%        2   &   333.7&   0.1 &   79.6    &   0.1\\
%        3   &   344.1&   0.1 &   79.7    &   0.1\\
%        4   &   363.7&   0.1 &   80.0    &   0.1\\
%        5   &   383.4&   0.1 &   80.1    &   0.1\\
%        \hline
%    \end{tabular}
%    \par}
%    \caption{Lunghezza della corda in funzione delle masse appese}
%\end{table}


%Sample: Multiline values
%\begin{align*}
%    &M_{tot} = (20.6 \pm 0.1) \ \text{g} \\
%    &m_{g} = (3.2 \pm 0.1) \ \text{g} \\
%    &L_{r,tot} (77.5 \pm 0.1) \ \text{cm} \\
%    &L_{r} = (75.8 \pm 0.1) \ \text{cm} \\
%\end{align*}

%Sample: Single line value
%\[
%    L_{0}=(78.8 \pm 0.1) \ \text{cm}
%\]


%Sample: Chart
%\begin{center}
%\begin{tikzpicture}
%    \begin{axis}[
%        title=Determinazione costante elastica,
%        xlabel=$L \text{ (m)}$,
%        ylabel=$M_{s} \text{ (kg)}$,
%        minor y tick num=1,
%        minor x tick num=1,
%        x tick label style={
%            /pgf/number format/.cd,
%            fixed,
%            fixed zerofill,
%            precision=3,
%            /tikz/.cd
%        }
%    ]
%        \addplot[color=blue, domain=0.794:0.802]{x*9.079-6.893};
%        \addplot+[scatter, only marks, mark=o, error bars/.cd, y dir=both,y explicit, x dir=both, x explicit, error %mark=-]
%            coordinates {
%                (0.7950,0.3242)   +-  (0.001,0.0001)
%                (0.7960,0.3337)   +-  (0.001,0.0001)
%                (0.7970,0.3441)  +-  (0.001,0.0001)
%                (0.8000,0.3637)  +-  (0.001,0.0001)
%                (0.8010,0.3834)  +-  (0.001,0.0001)
%        };
%    \end{axis}
%\end{tikzpicture}
%\end{center}
%Raccolta dati misurati sotto forma di tabella ordinata
%Presentare non solo i dati elaborati ma anche quelli grezzi
%(questi ultimi, eventualmente, in appendice).
%Presentare eventuali grafici con gli assi correttamente definiti (unità di 
%misura, cifre significative, identificativo assi).

\subsection{Charge to mass ratio of the electron}
The intitial setup is realized by connecting the following components:
\begin{itemize}
	\item Helmholtz coil to direct current (DC) power source that induce current intensity $I$;
	\item capacitor plates to direct current (DC) power source that induce a voltage $\Delta V$;
	\item filament that produce electrons to alternate current (AC) power source.
\end{itemize}

The measurement of $I$ and $\Delta V$ is performed by a pair of digital multimeter connected 
respectively in series and in parallel of their circuits.

\subsubsection{Equipment orthogonal to Earth's magnetic field}
%In this experience the equipment is placed orthogonal to the Earth's magnetic field, 
%this configuration is realized by placing the equipment in a way that the direction 
%of the magnetic field produced by the pair of Helmholtz coil is orthogonal to the Earth's magnetic field. 
The implementation of this configuration in the experience is qualitative, 
meaning that the direction of the Eart's magnetic field is approximately established 
and the equipment is placed according.
This approximation increase the expected systematic error introduced by the horizontal component of 
the Earth's magnetic field.



\subsubsection{Equipment parallel to Earth's magnetic field}

\subsubsection{Equipment counter parallel to Earth's magnetic field}

\subsection{Horizontal component's intensity of Earth's magnetic field}
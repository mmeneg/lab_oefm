%documentMetadata{}
\documentclass[draft, a4paper,12pt]{article}

%\usepackage[italian]{babel}
\usepackage{pgfplots}
\usepackage{amsmath}
\usepackage{float}
\usepackage{graphicx} %for the scheme
\usepackage{array}
\usepackage[margin=1in]{geometry}
\pgfplotsset{width=10cm,compat=1.9}

\begin{document}

\parskip=10pt plus 1pt
\parindent=0pt

\begin{center}
    Universita' degli studi di Milano

    Corso di Laurea Triennale in Fisica

    \par{\Large{Laboratorio di Fisica con Elementi di Statistica}}

    Relatore: M. Meneghini

    5 Maggio 2024
\end{center}

\par\noindent\rule{\textwidth}{0.4pt}

\begin{abstract}
    \begin{center}
    \par{\large{Lab report: Measuring the charge-to-mass ratio of the electron}}

    Text here
    
    \end{center}
\end{abstract}

\par\noindent\rule{\textwidth}{0.4pt}

%Scopo e modalità (in breve) 
%Descrivere sinteticamente ciò che ci si propone di fare 
%nell'esperimento, ossia cosa si vuole misurare, verificare, provare… Cosa 
%si dovrà misurare? Con che strumento? Potete inserire foto o schemi.
\section{Objectives}

The purpose of this experiment is to determine the speed of light by employing a Foucault rotating-mirror setup. A laser beam is directed onto a rapidly spinning mirror, then reflected toward a concave mirror and back. Because light requires a finite time to travel, any increase in the rotating mirror’s angular velocity causes a small lateral shift of the returning beam, which is observable in the microscope’s field of view. By systematically varying the mirror’s rotation speed and measuring the corresponding spot displacement, it becomes possible to calculate the speed of light from macroscopic, measurable quantities such as the rotation frequency, the geometrical distances along the optical bench (\(D, a, f_2\)), and the shift observed in the microscope. The precise alignment of the beam and mirrors is crucial, so considerable attention is given to centering the laser beam on each element and ensuring the return beam follows the same path. 


The data collection includes several measurements of the spot’s displacement  \(\Delta\delta\) at different rotational frequencies in both clockwise and counterclockwise directions, as well as transitions from maximum speed in one direction to maximum speed in the other. This set of measurements, along with the known distances and lens focal lengths, allows for calculating the speed of light and estimating uncertainties arising from micrometer resolution, alignment accuracy, and possible variations in the mirror’s angular velocity.


\begin{figure}[h]
    \centering
    \includegraphics[width=\linewidth]{LightSpeed/image123.png}
    \caption{Foucault device used in the laboratory experiment}
    \label{fig:foucault}
\end{figure}


%Raccolta dati misurati sotto forma di tabella ordinata
%Presentare non solo i dati elaborati ma anche quelli grezzi
%(questi ultimi, eventualmente, in appendice).
%Presentare eventuali grafici con gli assi correttamente definiti (unità di 
%misura, cifre significative, identificativo assi).
\section{Procedure}
%Raccolta dati misurati sotto forma di tabella ordinata
%Presentare non solo i dati elaborati ma anche quelli grezzi
%(questi ultimi, eventualmente, in appendice).
%Presentare eventuali grafici con gli assi correttamente definiti (unità di 
%misura, cifre significative, identificativo assi).

\subsection{Charge to mass ratio of the electron}
The intitial setup is realized by connecting the following components:
\begin{itemize}
	\item Helmholtz coils to direct current (DC) power source that induce current intensity $I$;
	\item capacitor plates to direct current (DC) power source that induce a voltage $\Delta V$;
	\item filament that produce electrons to alternate current (AC) power source.
\end{itemize}

The measurement of $I$ and $\Delta V$ are performed by a pair of digital multimeter connected 
respectively in series and in parallel of their circuits.

The measure of the distance between the two Helmholtz coils is done with a slide gauge, by taken 
the internal distance $d_{\text{int}}$ and external distance $d_{\text{ext}}$ between the coils.
The distance between the two coils $d$ is calculated as:
\[
    d=\frac{(d_{\text{int}} + d_{\text{ext}})}{2}
\]

The following table represent the measures $d_{\text{int}}$, $d_{\text{ext}}$, and the calculate $d$:
\begin{table}[!htbp]
    {\par\centering
    \begin{tabular}{cccc}
        \hline
        Measure & $d_{\text{int}} \text{ (cm)}$ & $d_{\text{ext}} \text{ (cm)}$ & $d$ \text{ (cm)}\\
        \hline
        1   &   13.260& 17.740&   15.500\\
        2   &   13.342& 17.786&   15.564\\
        3   &   13.310& 17.780&   15.545\\
        \hline
    \end{tabular}
    \par}
    \caption{External and internal distance between Helmholtz coils}
\end{table}

The configuration of the Helmholtz coils is such that the distance between the Helmholtz coils $d$ is 
the radius of the coils $R_b$:
\[
    R_b=15.536 \pm 0.268 \ \text{cm}
\]

\subsubsection{Equipment orthogonal to Earth's magnetic field}
%In this experience the equipment is placed orthogonal to the Earth's magnetic field, 
%this configuration is realized by placing the equipment in a way that the direction 
%of the magnetic field produced by the pair of Helmholtz coil is orthogonal to the Earth's magnetic field. 
The implementation of this configuration in the experience is qualitative, 
meaning that the direction of the Eart's magnetic field is approximately established 
and the equipment is placed accordingly.

A certain voltage $\Delta V$ is set on the capacitor plates (measured by digital multimeter in parallel), then a proper current intensity $I$ (measured by a digital multimeter in series)
is set on the Helmholtz coils to change the value of the magnetic field $B_z$, this modify 
the trajectory of the electrons in a way the the shape of the electrons beam is a circle.

The measure of the diameter of the circle is performed by placing a pair of sliders that can freely move 
on a rail that cross the center of the circle.
A slide gauge is used to measure the distance between the sliders $D$, two measures $D_1$ and $D_2$ are performed for each configuration of $\Delta V$ and $I$.

The following table represent the two measures of the distance bewteen the sliders $D_1$, $D_2$, the radius of the circle $R$ calcualted as $R=D/2$ where $D$ is the semisum of the distances $D=(D_1 + D_2)/2$, and the related value of $\Delta V$ and $I$. 
%This approximation increase the expected systematic error introduced by the horizontal component of 
%the Earth's magnetic field.

\begin{table}[!htbp]
    {\par\centering
    \begin{tabular}{cccccc}
        \hline
        Measure & $I \text{ (A)}$ & $\Delta V \text{ (V)}$ & $D_1 \text{ (cm)}$ & $D_2 \text{ (cm)}$ & $R \text{ (cm)}$\\
        \hline
        1   &   1.133&   162.5&   9.374&   9.190& 4.641\\
        2   &   1.187&   176.8&   9.120&   9.200& 4.580\\
        3   &   1.160&   195.5&  10.170&  10.100& 5.068\\
        4   &   1.375&   213.8&   9.044&   9.048& 4.523\\
        5   &   1.307&   224.5&   9.780&   9.778& 4.890\\
        6   &   1.554&   242.2&   8.550&   8.580& 4.283\\
        7   &   1.522&   257.8&   9.072&   9.040& 4.528\\
        8   &   1.607&   274.8&   8.980&   9.010& 4.498\\
        9   &   1.740&   295.9&   8.520&   8.522& 4.261\\
        10  &   1.240&   200.0&   9.550&   9.560& 4.778\\
        \hline
    \end{tabular}
    \par}
    \caption{Orthogonal configuration - Measure of electrons beam diameter for each configuration, and related radius}
\end{table}

\subsubsection{Equipment parallel to Earth's magnetic field}
The equipment is then rotated by $90^{\circ}$ and used the same procedure as previously explained.

The following table shows the measures and the calculated radius $R$
\begin{table}[!htbp]
    {\par\centering
    \begin{tabular}{cccccc}
        \hline
        Measure & $I \text{ (A)}$ & $\Delta V \text{ (V)}$ & $D_1 \text{ (cm)}$ & $D_2 \text{ (cm)}$ & $R \text{ (cm)}$\\
        \hline
        1   &   1.133&   149.2&   8.510&   8.450& 4.240\\
        2   &   1.367&   166.7&   7.790&   7.770& 3.890\\
        3   &   1.293&   179.4&   8.680&   8.650& 4.333\\
        4   &   1.428&   193.1&   8.290&   8.320& 4.153\\
        5   &   1.532&   209.0&   7.940&   7.944& 3.971\\
        6   &   1.552&   226.1&   8.360&   8.362& 4.181\\
        7   &   1.657&   240.8&   8.010&   8.000& 4.003\\
        8   &   1.353&   257.7&  10.292&  10.260& 5.138\\
        9   &   1.328&   272.6&  10.790&  10.800& 5.398\\
        10  &   1.372&   292.2&  10.608&  10.620& 5.307\\
        \hline
    \end{tabular}
    \par}
    \caption{Parallel configuration - Measure of electrons beam diameter for each configuration, and related radius}
\end{table}


\subsubsection{Equipment counter parallel to Earth's magnetic field}
The equipment is then rotated by $-180^{\circ}$ and used the same procedure as previously explained.

The following table shows the measures and the calculated radius $R$:
\begin{table}[!htbp]
    {\par\centering
    \begin{tabular}{cccccc}
        \hline
        Measure & $I \text{ (A)}$ & $\Delta V \text{ (V)}$ & $D_1 \text{ (cm)}$ & $D_2 \text{ (cm)}$ & $R \text{ (cm)}$\\
        \hline
        1   &   0.811&   151.6&  11.280&  11.300& 5.645\\
        2   &   0.881&   181.2&  12.290&  12.400& 6.173\\
        3   &   1.038&   212.6&  11.650&  11.620& 5.818\\
        4   &   1.151&   242.6&  10.280&  10.190& 5.118\\
        5   &   1.289&   272.7&  10.480&  10.470& 5.238\\
        6   &   1.302&   296.9&  11.130&  11.111& 5.560\\
        7   &   1.204&   259.4&  11.350&  11.200& 5.638\\
        8   &   1.136&   228.2&  10.980&  10.990& 5.493\\
        \hline
    \end{tabular}
    \par}
    \caption{Antiparallel configuration - Measure of electrons beam diameter for each configuration, and related radius}
\end{table}


\subsection{Horizontal component's intensity of Earth's magnetic field}
The intitial setup is realized by connecting the Helmholtz coils to direct current (DC) power source that induce current intensity $I$.
The measurement of $I$ is performed by a pair of digital multimeter connected in series.
A switch is used to revert the flow of the current to the Helmholtz coils, this simplify the measurement of negative 
angle deflections.

The current $I$ is changed to obtain a certain deflection $\theta $ of the magnetic needle from the rest position.

The following table shows the measurement of $\theta$ respect to $I$
\begin{table}[!htbp]
    {\par\centering
    \begin{tabular}{ccc}
        \hline
        Measure & $I \text{ (mA)}$ & $\theta$ \\
        \hline
        1   &    14.33&    $45^{\circ}$\\
        2   &    15.76&   $-45^{\circ}$\\
        3   &    17.30&   $ 50^{\circ}$\\
        4   &    20.20&   $-50^{\circ}$\\
        5   &    21.38&   $ 55^{\circ}$\\
        6   &    22.60&   $-55^{\circ}$\\
        7   &    26.55&   $ 60^{\circ}$\\
        8   &    26.29&   $-60^{\circ}$\\
        9   &    34.98&   $ 65^{\circ}$\\
        10  &    33.65&   $-65^{\circ}$\\
        \hline
    \end{tabular}
    \par}
    \caption{Measure of magnetic needle deflection}
\end{table}


%Stima degli errori delle grandezze misurate (giustificare il metodo scelto)
%Valore finale della grandezza da determinare ed incertezza
%Ricordarsi di dare i valori con i rispettive errori e attenzione alle cifre significative! 
\section{Data Analysis}
I dati che verranno di seguito analizzati sono stati raccolti, seguendo la medesima procedura descritta precedentemente, in un due momenti differenti dai seguenti gruppi:
\begin{itemize}
	\item Gruppo A: il 6 Novembre 2024, da S. Pini e V. Malinouskaya
	\item Gruppo B: il 12 Novembre 2024, da A. Spagnolo e M. Meneghini
\end{itemize}

\subsection{Gruppo A}
Di seguito sono riportate le misure degli angolo $\theta_1$ e $\theta_2$:
\begin{table}[!htbp]
    {\par\centering
    \begin{tabular}{ccccc}
        \hline
        Misura & $\theta_1 \text{ ($^{\circ}$)}$ & $\theta_1 \text{ ($^{\prime}$)}$ & $\theta_2 \text{ ($^{\circ}$)}$ & $\theta_2 \text{ ($^{\prime}$)}$ \\
        \hline
        1   &	354	&	20.0	&	234	&	20.0\\
        2	&	348	&	57.0	&	228	&	58.0\\
        3	&	353	&	19.0	&	233	&	19.0\\
        4	&	352	&	27.0	&	232	&	25.0\\
        5	&	349	&	2.0	&	229	&	5.0\\
        6	&	354	&	22.0	&	234	&	28.0\\
        7	&	359	&	40.0	&	239	&	40.0\\
        8	&	359	&	45.0	&	239	&	43.0\\
        9	&	352	&	20.0	&	232	&	18.0\\
        10	&	355	&	5.0	&	235	&	6.0\\
        \hline
    \end{tabular}
    \par}
    \caption{Gruppo A - misura degli angoli $\theta_1$ e $\theta_2$}
\end{table}

%Ogni misura degli angoli $\theta_1$ e $\theta_2$ ha una incertezza pari $0.5^{\prime}$, in gradi decimali $0.008^{\circ}$.

L'angolo $\alpha$ e' ottenuto tramite la formula:
\[
	\alpha = 180^{\circ} - \Delta \theta
\]

Di seguito i valori di $\Delta \theta$ e $\alpha$:
\begin{table}[!htbp]
    {\par\centering
    \begin{tabular}{ccccc}
        \hline
        Misura & $\Delta \theta \text{ ($^{\circ}$)}$ & $\Delta \theta \text{ ($^{\prime}$)}$ & $\alpha \text{ ($^{\circ}$)}$ & $\alpha \text{ ($^{\prime}$)}$\\
        \hline
        1   &   120 &   0.0     &   60  &   0.0\\
        2   &   119 &   59.0    &   60  &   1.0\\
        3   &   120 &   0.0     &   60  &   0.0\\
        4   &   120 &   2.0     &   59  &   58.0\\
        5   &   119 &   57.0    &   60  &   3.0\\
        6   &   119 &   54.0    &   60  &   6.0\\
        7   &   120 &   0.0     &   60  &   0.0\\
        8   &   120 &   2.0     &   59  &   58\\
        9   &   120 &   2.0     &   59  &   58\\
        10  &   119 &   59.0    &   60  &   1.0\\
        \hline
    \end{tabular}
    \par}
    \caption{Gruppo A - angoli $\Delta \theta$ e $\alpha$ calcolati}
\end{table}

Il miglior valore di $\alpha$ e' ottenuto tramite la media delle dieci misurazioni indipendenti effettuate, mentre l'errore $\sigma_{\alpha}$ e' ottenuto come deviazione standard della media,
\[
    \alpha = 60^{\circ} \ 0.5^{\prime} \pm 0.8^{\prime}
\]
%in gradi decimali
%\[
%    \alpha = (60.01 \pm 0.01)^{\circ}
%\]

Di seguito sono riportate le misure dell'angolo $\theta_0$ relativo alla posizione del cannocchiale rispetto al collimatore con il raggio non deviato:
\begin{table}[!htbp]
    {\par\centering
    \begin{tabular}{ccc}
        \hline
        Misura & $\theta_0 \text{ ($^{\circ}$)}$ & $\theta_0 \text{ ($^{\prime}$)}$ \\
        \hline
        1   &   238 &   52.0\\
        2   &   238 &   53.0\\
        3   &   238 &   53.0\\
        4   &   238 &   53.0\\
        5   &   238 &   51.0\\
        6   &   238 &   54.0\\
        7   &   238 &   52.0\\
        8   &   238 &   53.0\\
        9   &   238 &   53.0\\
        10  &   238 &   52.0\\
        \hline
    \end{tabular}
    \par}
    \caption{Gruppo A - misura angolo $\theta_0$}
\end{table}

Il miglior valore di $\theta_0$ e' ottenuto tramite la media delle dieci misurazioni indipendenti effettuate, mentre essendo la deviazione standard della media $\sigma_{\bar{\theta_0}}$ inferiore alla sensibilita' del nonio, viene preso come errore $\sigma_{\theta_0}$ il doppio della deviazione standard della media $2\sigma_{\bar{\theta_0}}$,
\[
    \theta_0 = 238^{\circ} \ 52.6^{\prime} \pm 0.5^{\prime}
\]
a significare che il valore vero di $\theta_0$ ha il 95.5\% di probabilita' di essere contenuto in tale intervallo.

Di seguito e' riportata la misura dell'angolo $\theta_{\lambda}$ in corrispondenza di una determinata lunghezza d'onda $\lambda$:
\begin{table}[!htbp]
    {\par\centering
    \begin{tabular}{ccc}
        \hline
        $\lambda \text{(nm)}$ & $\theta_{\lambda} \text{ ($^{\circ}$)}$ & $\theta_{\lambda} \text{ ($^{\prime}$)}$ \\
        \hline
        404.7   &   130 &   46.0\\
        496.0   &   127 &   42.0\\
        546.1   &   126 &   3.0\\
        577.0   &   125 &   13.0\\
        NA   &   123 &   8.0\\
        \hline
    \end{tabular}
    \par}
    \caption{Gruppo A - misura angolo $\theta_{\lambda}$ in funzione di $\lambda$}
\end{table}

L'incertezza $\sigma_{\theta_{\lambda}}$ sulla singola misura di $\theta_{\lambda}$ e' stata stimata pari a $1.0^{\prime}$ per la difficolta' nell'individuare nitidamente il punto esatto di inversione.

Il valore dell'angolo di deviazione minima $\delta_{\lambda}$ e' ottenuto tramite la relazione:
\[
    \delta_{min} (\lambda) = |\theta_{\lambda} - \theta_0|
\]

mentre l'incertezza $\sigma_{delta}$ e' pari alla somma in quadratura di $\sigma_{\theta_{\lambda}}$ e $\sigma_{\theta_0}$.

Di seguito e' riportato il valore calcolato di $\delta_{min}$ e la sua incertezza $\sigma_{delta}$ in funzione di $\lambda$:
\begin{table}[!htbp]
    {\par\centering
    \begin{tabular}{cccc}
        \hline
        $\lambda \text{(nm)}$ & $\delta_{min} \text{ ($^{\circ}$)}$ & $\delta_{min} \text{ ($^{\prime}$)}$ & $\sigma_{delta} \text{ ($^{\prime}$)}$\\
        \hline
        404.7   &   108 &   6.6  & 1.1\\
        496.0   &   111 &   10.6 & 1.1\\
        546.1   &   112 &   49.6 & 1.1\\
        577.0   &   113 &   39.6 & 1.1\\
        NA   &   115 &   44.6 & 1.1\\
        \hline
    \end{tabular}
    \par}
    \caption{Gruppo A - angolo $\delta_{min}$ in funzione di $\lambda$}
\end{table}

Il valore di $\eta (\lambda)$ e' stato calcolato tramite la relazione:
\[
    \eta (\lambda) = \frac{\sin{\frac{\alpha + \sigma_{min} (\lambda)}{2}}}{\sin{\frac{\alpha}{2}}}
\]

mentre l'errore $\sigma_{\eta}$ e' stato calcolato tramite la formula di propagazione degli errori, ovvero la relazione:
\[
    %\sigma_{\eta} = \sqrt{\left( \frac{1}{2 \sin^2{(\frac{\alpha}{2}})} \left( \cos{\left(\frac{\alpha+\sigma_{min}}{2} \right)}\sin{\left(\frac{\alpha}{2}\right)} - \sin{\left( \frac{\alpha+\sigma_{min}}{2} \right)}\cos{\left(\frac{\alpha}{2}\right)} \right) \sigma_{\alpha} \right)^2 + \left( \frac{1}{2} \frac{\cos{\left(\frac{\alpha+\sigma_{min}}{2}}{\sin{\left(\frac{\alpha}{2}\right)}} \sigma_{\lambda} \right)^2}
    \sigma_{\eta} = \left| {\frac{\cos{\left( \frac{\alpha + \sigma_{min}}{2} \right)}}{2 \sin{\left(\frac{\alpha}{2}} \right)}} \right | \sqrt{\left( 1-\frac{\tan{\left(\frac{\alpha + \sigma_{min}}{2}}\right)}{\tan{\left( \frac{\alpha}{2} \right)}} \right)^2 \sigma_{\alpha}^2 + \sigma_{\delta}^2}
\]

Di seguito

\subsection{Gruppo B}
Di seguito sono riportate le misure degli angolo $\theta_1$ e $\theta_2$:
\begin{table}[!htbp]
    {\par\centering
    \begin{tabular}{ccccc}
        \hline
        Misura & $\theta_1 \text{ ($^{\circ}$)}$ & $\theta_1 \text{ ($^{\prime}$)}$ & $\theta_2 \text{ ($^{\circ}$)}$ & $\theta_2 \text{ ($^{\prime}$)}$ \\
        \hline
        1   &   246 &   30.0    &   7 &   30.0\\
        2   &   249 &   0.0    &   9 &   0.0\\
        3   &   256 &   42.0    &   16 &   41.0\\
        4   &   253 &   29.0    &   12 &   27.0\\
        5   &   242 &   0.0 &   2 &   0.0\\
        \hline
    \end{tabular}
    \par}
    \caption{Gruppo B - misura degli angoli $\theta_1$ e $\theta_2$}
\end{table}

L'angolo $\alpha$ e' ottenuto tramite la formula:
\[
    \alpha = 180^{\circ} - \Delta \theta
\]

%Confronto con i dati noti e valutazione di discrepanze.
\section{Conclusion}
Per il gruppo A, l'errore nella presa dati dell'angolo di deviazione minima $\delta_{min}$ non ha permesso di compiere ulteriori analisi.
Questo errore può essere derivato da un errore nella procedura eseguita nelle misurazioni di $\theta_0$ e di $\theta_{\lambda}$.

Per il gruppo B, l'andamento approssimato di Cauchy tra indice di rifrazione $\eta$ e lunghezza d'onda $\lambda$ e' rispettato con una buona approssimazione.
I valori del $\chi^2$ ci suggeriscono che le misure di $\delta_{min}$ della riga "giallo sx" possano contenere un errore sistematico probabilmente dovuto ad una difficolta' nella visualizzazione della riga a causa della vicinanza dell'altra riga "giallo dx".

Considerati gli errori $\sigma_{\lambda}$ delle lunghezze d'onda $\lambda$ misurate nell'esperienza "spettromentro a reticolo", si e' deciso di non propagare tali errori ai parametri del fit poiche' di un ordine di grandezza inferiore all'errore $\sigma_{\eta^2}$ e quindi trascurabili.

La buona correlazione, anche se migliorabile, tra la distribuzione osservata e la relazione di Cauchy ci garantisce che gli errori sistematici propagati siano abbastanza piccoli.

\end{document}

\par\noindent\rule{\textwidth}{0.4pt}

\begin{abstract}
    \begin{center}
    %\par{\large{Lab report: Measuring the charge-to-mass ratio of the electron}}
    We calculated the electron's the e/m ratio by measuring the radius of a deflected electron beam in a fixed magnetic field to vary current intensity and tension. The values were found to be $\frac{e}{m}=(1.67\pm0.01)\, \mathrm{C/Kg}$ for the orthogonal setup, $\frac{e}{m}=(1.66\pm0.01)\, \mathrm{C/Kg}$ for the parallel one and $\frac{e}{m}=(1.68\pm0.01)\, \mathrm{C/Kg}$ for the antiparallel one, which doesn't correlates with the expected value of $ \frac{e}{m}=1.76\,\,\, \mathrm{C/Kg}$. We then calculated the horizontal component's intensity of Earth's magnetic field using the same coils and a magnetic needle. We found a value of $B_{tf}=(24.08\pm0.56)\,\, \mu\mathrm{T}$ which is consistent (??) with the accepted value of $20-40\,\,\mu\mathrm{T}$.
    
    
    \end{center}
\end{abstract}

\par\noindent\rule{\textwidth}{0.4pt}